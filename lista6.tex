\documentclass{article}
\usepackage[utf8]{inputenc}
\usepackage{amsmath}
\usepackage{Sweave}
\begin{document}
\Sconcordance{concordance:lista6.tex:lista6.Rnw:%
1 3 1 1 0 42 1}


\section*{Exercise 6.11}

\begin{itemize}
\item[(a)] O modelo nao-parametrico possui a seguinte equação para y.

\begin{equation}
g(x)=\sum\frac{\Phi(\frac{|x-x_{n}|}{\sigma})}{\sum\Phi(\frac{|x-x_{i}|}{\sigma})}y_{n}
\end{equation}
Logo, quando $|x| \to \infty$, o kernel gaussiando vai para zero fazendo o valor de g(x) ir para zero.


O parametrico possui o seguinte comportamento,

\begin{align*}
h(x)&=\sum w_{n}z_{in}+w_{0}+\epsilon_{i} \\
&\rightarrow Y=\hat{Z}w+\epsilon \\
&\to \tilde{w}=(\tilde{Z}^{T}\overline{Z})Z^{T}y
\end{align*}

Agora quando $|x| \to \infty$, o valor da saida do kernel também vai para zero, porém, o modelo linear possui o valor de intercpeto, que não é multiplicado pela função kernel, o que faz do resultado da hipótese ser igual ao valor do intercepto.

\item[(b)] Quando $Z$ é invertível, os parâmetros da hipótese para y se descrevem da seguinte forma:

\begin{align*}
Y&=\hat{Z}w+\epsilon \\
\hat{Z}^{-1}Y&=\hat{Z}^{-1}\hat{Z}w\rightarrow w=\hat{Z}^{-1}Y
\end{align*}

Com o  $E_{in}$ 

\begin{align*}
Y-\hat{Y}&=Y-\hat{Z}\hat{Z}^{-1}Y\\ 
&\rightarrow Y-\hat{Y}=Y-IY \\ 
&\rightarrow Y-\hat{Y}=0
\end{align*}

\end{itemize}


\end{document}
